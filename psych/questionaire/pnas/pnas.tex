\documentclass[a4paper,10pt]{article}


\usepackage[text={6in,10in},centering]{geometry}.
\usepackage{xepersian}
\setdigitfont[Scale=1]{Yas}
\setmainfont[Script=Arabic,Mapping=farsidigits]{Yas}


\begin{document}

 این پرسش‌نامه شامل تعدادی واژه است که احساسات و عواطف مختلف را نشان می‌دهند. هر مورد را بخوانید و سپس در زیر پاسخ مناسب علامت بگذارید.
بیان کنید که طی هفته گذشته تا چه حد آن حس را تجربه کرده‌اید.
\vspace{0.5cm}
\hrule
\vspace{0.25cm}
\begin{center}
\textbf{جنس \ldots \ldots \ldots سن \ldots \ldots میزان تحصیلات \ldots \ldots \ldots} 
\end{center}
\vspace{0.25cm}
\hrule
\vspace{0.25cm}

% مصمم & خیلی کم یا اصلا & کمی & در حد متوسط & به مقدار قابل ملاحظه & در حد بسیار زیاد\\


\begin{center}
\begin{tabular}{|r|c|c|c|c|c|}
\hline
 & خیلی کم یا اصلا & کمی & در حد متوسط & به مقدار قابل ملاحظه & در حد بسیار زیاد\\
\hline
تحریک‌پذیر &  &  &  &  & \\
\hline
هوشیار &  &  &  &  & \\
\hline
شرمنده &  &  &  &  & \\
\hline
پرشور &  &  &  &  & \\
\hline
عصبی &  &  &  &  & \\
\hline
مصمم &  &  &  &  & \\
\hline
دقیق &  &  &  &  & \\
\hline
ترسان &  &  &  &  & \\
\hline
سرزنده &  &  &  &  & \\
\hline
ترسیده &  &  &  &  & \\
\hline
علاقه‌مند &  &  &  &  & \\
\hline
آشفته &  &  &  &  & \\
\hline
هیجان‌زده &  &  &  &  & \\
\hline 
ناراحت &  &  &  &  & \\
\hline
قوی &  &  &  &  & \\
\hline
احساس تقصیر و گناه &  &  &  &  & \\
\hline
وحشت‌زده &  &  &  &  & \\
\hline
میل به خشونت &  &  &  &  & \\
\hline
مشتاق &  &  &  &  & \\
\hline
احساس افتخار و سرافرازی &  &  &  &  & \\
\hline

\end{tabular}
\end{center}


\end{document}
