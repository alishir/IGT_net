\documentclass[a4paper,10pt]{article}
%\documentclass[a4paper,10pt]{scrartcl}

\usepackage{graphicx}
\usepackage{subfig}
\usepackage{hyperref}
\usepackage{arydshln}

\usepackage{eso-pic}
\usepackage{fontspec}
\newfontfamily\rmfamily[Mapping=tex-text]{Yas}
\usepackage[RTLdocument]{bidi}
\usepackage{xepersian}
\setdigitfont[Scale=1]{Yas}
\setmainfont[Script=Arabic,Mapping=farsidigits]{Yas}
\setlatintextfont{Yas}
\usepackage[text={5in,8in},centering]{geometry}
\linespread{1.8}

\title{}
\author{}
\date{}

\begin{document}
با تشکر از همکاری و وقتی که برای انجام این آزمایش می‌گذارید.
بازی‌ای که شما انجام خواهید داد، یک کارت‌بازی عادی است.
در مقابل شما چهار دسته کارت قرار دارند که با حروف
 $A, B, C, D$
 مشخص شده‌اند.
این دسته کارت‌ها در اتاقی دیگر قرار دارند و شما تصویر آن‌ها را توسط دوربینی که در آن محل تعبیه شده مشاهده می‌کنید.
در هر بار، شما یک دسته کارت را با فشار دادن کلید متناظر آن دسته کارت از روی صفحه کلید انتخاب می‌کنید
(کلیدهای $A,B,C,D$)
، سپس اولین کارت روی آن دسته کارت، توسط فردی که در آن اتاق
حضور دارد برگردانده می‌شود.
با انتخاب هر کارت، شما به مقداری که پشت آن نوشته شده امتیاز کسب می‌کنید، اما کارت‌هایی هم هستند که علاوه بر دریافت امتیاز 
ممکن است امتیاز هم از دست بدهید. در حالتی که امتیاز کسب می‌کنید، عبارتی مشابه
$\mbox{ You won }X\mbox{ score }$
را مشاهده می‌کنید، که 
$X$
میزان امتیاز کسب شده است.
در حالتی که هم امتیاز کسب می‌کنید و هم امتیاز از دست می‌دهید، عبارتی مشابه
$\mbox{ You won }X \mbox{ score but lost }Y }$
مشاهده می‌کنید، که 
$X$
بیان‌گر امتیاز کسب شده و 
$Y$
بیان‌گر امتیازی است که از دست داده‌اید.
شما در این بازی باید سعی کنید بیشترین امتیاز ممکن را کسب کنید.

لطفا پس از اینکه دایره سیاه رنگ از وسط صفحه محو شد، بازی را شروع کنید و تا مشاهده مجدد آن به بازی ادامه دهید.
اگر در حین انجام بازی، کارت‌های دسته‌ای تمام شد، به بازی ادامه دهید، تا وقتی که دایره سیاه ظاهر شود.
شما در انجام این بازی هیچ‌گونه محدودیت زمانی ندارید.
در پایان مبلغی معادل ضریبی از امتیازی که کسب کرده‌اید به شما تعلق خواهد گرفت. 


\end{document}
