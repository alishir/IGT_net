\documentclass{article}
\usepackage{graphicx}
\usepackage{subfig}
\usepackage{hyperref}

\usepackage{eso-pic}
\usepackage{fontspec}
\newfontfamily\rmfamily[Mapping=tex-text]{Yas}
\usepackage[RTLdocument]{bidi}
\usepackage{fancyhdr}



\pagestyle{fancy}
%\textheight 20cm

\usepackage{xepersian}
\setdigitfont[Scale=1]{Yas}
\setmainfont[Script=Arabic,Mapping=farsidigits]{Yas}
\setlatintextfont{Yas}
\settextfont{Yas}


%\oddsidemargin 0cm
\topmargin -0.5cm
\headsep 3cm

\chead
{
بسمه تعالی \\
\vspace{.5cm}
\textbf{شبکه‌های اجتماعی} \\
\vspace{.3cm}
تمرین شماره 2 \\
}
\rhead
{
 \includegraphics[scale=.1]{./logo/ut_logo.png}
}

\lhead
{
\includegraphics[scale=.1]{./logo/fanni_logo.jpg}
}

\begin{document}
\section{روش}
در این قسمت به بیان روشی که برای مقایسه تاثیر هر دسته آزمایش  بر رفتار افراد در بازی می‌پردازیم.
در ابتدا افرادی که در هر گروه به صورت رندم بازی کرده بودند را از دیگر افراد جدا می‌کنیم. فرآیندی که برای این جداسازی طی می‌کنیم به این صورت است که 
تعدادی رفتار رندم به هر گروه اضافه می‌کنیم، سپس اعضاء گروه را بر اساس رفتاری که در انتخاب دسته‌ کارت‌ها از خود نشان داده‌اند سه دسته می‌کنیم.

در ابتدا رفتار افراد را با گروه فرضی که به صورت رندم بازی کرده‌اند بررسی می‌کنیم، این بررسی به این صورت انجام می‌شود که 
تعدادی بازیکن که به صورت رندم بازی کرده‌اند را به هر گروه اضافه می‌کنیم، سپس داده‌ها حاصل را در سه دسته، دسته‌بندی می‌کنیم،
به این امید که دسته‌ها شامل افرادی که بازی را یاد گرفته‌اند، افرادی که به صورت رندم عمل کرده‌اند و افرادی که 
در بازی کارآیی ضعیفی از خود نشان داده‌اند، تقسیم شوند.

پس از این تقسیم‌بندی، دسته‌ای که میانگین بیش‌تری دارند را به عنوان بهترین گروه در نظر می‌گیریم.
سپس به بررسی تعداد بازی‌کن‌های رندم در آن دسته می‌پردازیم، اگر 
این تعداد قابل توجه باشد، بیان‌گر این نکته است که اکثریت افراد در آن گروه بازی را یاد نگرفته‌اند
 و کارآیی آن‌ها در سطح رندم و پائین‌تر از آن بوده است.


همانطور که قبلا گفته شد، در این بازی، چند ویژگی برای دسته کارت‌ها وجود دارد. این ویژگی‌ها عبارت‌اند از :
\begin{itemize}
 \item نتیجه طولانی مدت
\item تکرار پاداش‌ها
\item تکرار خسران‌ها
\end{itemize}

هنگامی که از منظر نتیجه طولانی مدت کارآیی افراد را مورد ارزیابی قرار می‌دهیم،
  این نکته که افرادی که قلم و کاغذ در اختیار داشته‌اند
در سطح رندم رفتار کرده‌اند جلب توجه می‌کند.


\end{document}
